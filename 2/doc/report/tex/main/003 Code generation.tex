\section{Code generation}
 
For each LLVM IR operation we built a small code generating object. Once this object is created it (in most cases) immediatly generates the code he is supposed to.
Code is generated from the leaves of the AST (Abstract Syntactical Tree) to the root.

Since LLVM IR uses a lot of temporary variables, a \emph{VariableAllocator} object is used to create new variable names dynamically.
It's a very simple Variable object (containing a name and a type) that is passed from a lower to a higher level in the AST.
This way a recursive expression will see its most inner part resolved and replaced by a new temporary variable. Each expression finally resolves to a single variable as its code gets generated by the simple arithmetic code generation objects mentionned higher.

We do not keep the full AST in memory: the code is generated as soon as possible, as the scanner reads the file, this way we can compile very big files without using much memory.

