\subsubsubsection{DC Motor Operation}
    Torque is generated in DC motors from the magnetic force,
    also known as the Lorentz force, which is produced when an
    electric current is passed through a coil in a magnetic field.
    This force is given by \ref{eq:Force} named \nameref{eq:Force}.
    \begin{equation}
      F=q[E+(v\times B)]
      \label{eq:Force}
    \end{equation}
    where $F$ is the force perpendicular to the coil,
    $E$ is the electric field in the coil,
    $v$ is the velocity of the charged particles in the coil,
    and $B$ is the magnetic field. From mechanics, torque is
    \begin{equation}
      \tau=F\times r
      \label{eq:Torque}
    \end{equation}
    If the electrical force in \ref{eq:Force} is ignored,
    and the remaining magnetic force is used in \ref{eq:Torque},
    with the assumption that $v$ is perpendicular to $B$, we find that
    \begin{equation}
          \tau=qvBrsin\theta
      \label{eq:Magnetic}
    \end{equation}