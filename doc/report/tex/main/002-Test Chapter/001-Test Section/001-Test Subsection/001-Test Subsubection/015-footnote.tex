\subsubsubsection{Footnote}
L'efficacité de l'algorithme SURF est dû principalement à l'utilisation d'une représentation intermédiaire de l'image connue sous le nom d'\textit{image intégrale}.

L'image intégrale est une image numérique qui est calculée rapidement à partir de l'image originale. On l'utilise pour accélérer le calcul de chaque zone rectangulaire dont le sommet supérieur gauche est à l'origine de l'image.

Proposée en 1984\footnote{Crow, Franklin (1984). "\textit{Summed-area tables for texture mapping}". SIGGRAPH '84: \textit{Proceedings of the 11th annual conference on Computer graphics and interactive techniques}: 207–212} comme méthode d'\textit{infographie}, c'est en 2001 qu'elle a été reformulée par la méthode de Viola et Jones\footnote{Paul Viola et Michael Jones, \textit{Robust Real-time Object Detection} IJCV 2001}, dans le cadre de la \textit{vision par ordinateur}.