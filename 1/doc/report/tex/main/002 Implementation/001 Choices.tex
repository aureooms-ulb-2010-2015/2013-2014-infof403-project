\subsection{Choices}

In order to maintain efficiency or to avoid ambiguity we have added a certain number of constraints on the S-COBOL language. Here is the exhaustive list of our arbitrary choices and the reasons why we have made them.

\subsubsection{Signed numbers vs arithmetic expressions}

To avoid ambiguity an arithmetic expression will only be considered as one if a space is present between the operator and the operand. If no space can be found between the two, the token will be taken as an integer or a real.

\subsubsection{Comments}
\label{sec:impl:choices:comments}

Comments have to be on their own line (only space and tab characters are allowed). This way the / and * characters will not be seen as operators.

The RE for comments becomes:

\begin{verbatim}
^[ \t]*(\*|/).*\.\n;
\end{verbatim}

\subsubsection{Images vs Identifier}

We chose to make the parentheses in an image mandatory. This constraint can be made without loss of generality and ensure a more intuitive system:

\begin{itemize}
	\item Token 9 can be interpreted as an integer and is no longer cause of ambiguity.
	\item Tokens s9 and s9v9 can be used as identifiers while image representations s9(1) and s9(1)v9(1) still exist.  
\end{itemize}

Then the RE for Images becomes:

\begin{verbatim}
s?9(\([1-9][0-9]*\))(v9(\([1-9][0-9]*\)))?
\end{verbatim}

\subsubsection{ASCII only}

Our program will only recognise ASCII characters but could easily be extended to utf-8 charset.

\subsubsection{Keywords in lower case}

To ensure harmony our program will only accept lower case keywords. This allows the use of keywords as identifiers just by capitalizing one letter at least.






